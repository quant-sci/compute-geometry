\documentclass{article}
\usepackage[utf8]{inputenc}
\usepackage{geometry}
\geometry{a4paper, total={150mm,240mm}, top=25mm}
\usepackage{graphicx}
\usepackage{titling}
\usepackage{amsmath}
\usepackage{amsfonts}
\usepackage{amssymb}


\title{Lecture Notes on Computational Geometry}
\author{Kleyton da Costa}
\date{\today}
 
\usepackage{fancyhdr}
\fancypagestyle{plain}{%  the preset of fancyhdr 
    \fancyhf{} % clear all header and footer fields
    \fancyfoot[R]{\includegraphics[width=3cm]{di.png}}
    \fancyfoot[L]{\today}
    \fancyhead[L]{Lecture Notes on Computational Geometry}
    \fancyhead[R]{\theauthor}
}
\makeatletter
\def\@maketitle{%
  \newpage
  \null
  \vskip 1em%
  \begin{center}%
  \let \footnote \thanks
    {\LARGE \@title \par}%
    \vskip 1em%
    %{\large \@date}%
  \end{center}%
  \par
  \vskip 1em}
\makeatother

\begin{document}

\maketitle

\begin{center}
\begin{tabular}{@{}cc}
    \textbf{\theauthor}\\
    Department of Informatics\\
    PUC-Rio\\
    \today
\end{tabular}
\end{center}


\section{Introduction}

A área de geometria computacional surgiu nos anos de 1970 a partir do campo de estudos de projeto e análise de algoritmos. Podemos definir a geometria computacional como sendo o estudo sistemático de algoritmos e estruturas de dados para objetos geométricos, com foco em algoritmos que sejam assintóticamente rápidos. 

Dizemos que um subconjunto $\mathbb{S}$ do plano é \textit{convexo} se e somente se para cada par de pontos $p,q\in \mathbb{S}$ o segmento de reta $\overline{pq}$ está completamente contido em $\mathbb{S}$.

\subsection{Polígonos}

Uma maneira natural de representar um polígono é através de uma lista de vértices no sentido horário, começando a partir de um ponto qualquer. Assim, temos que um polígono é um conjutno de pontos $P = (p_{1}, p_{2},\dots, p_{n})$.

\end{document}
