\documentclass{article}
\usepackage[utf8]{inputenc}
\usepackage[brazilian]{babel}
\usepackage{geometry}
\geometry{a4paper, total={150mm,240mm}, top=25mm}
\usepackage{graphicx}
\usepackage{titling}
\usepackage{float}
\usepackage{natbib}
\usepackage{setspace}
\onehalfspacing

\title{Trabalho 2 - Diagrama de Voronoi}
\author{Kleyton da Costa (2312730)}
\date{\today}
 
\usepackage{fancyhdr}
\fancypagestyle{plain}{%  the preset of fancyhdr 
    \fancyhf{} % clear all header and footer fields
    \fancyfoot[R]{\includegraphics[width=3cm]{di.png}}
    \fancyfoot[L]{\today}
    \fancyhead[L]{Geometria Computacional}
    \fancyhead[R]{\theauthor}
}
\makeatletter
\def\@maketitle{%
  \newpage
  \null
  \vskip 1em%
  \begin{center}%
  \let \footnote \thanks
    {\LARGE \@title \par}%
    \vskip 1em%
    %{\large \@date}%
  \end{center}%
  \par
  \vskip 1em}
\makeatother

\begin{document}

\maketitle

\noindent\begin{tabular}{@{}ll}
    Aluno & \theauthor \\
    Professor &  Waldemar Celes (DI/PUC-Rio)
\end{tabular}

\section{Motivação}

Este trabalho tem como objetivo implementar um diagrama de voronoi a partir de um conjunto de pontos que não possui degenerações. 

\section{Metodologia}


\section{Experimentos}


\section{Considerações Finais}


\bibliographystyle{apa}
\bibliography{work2-voronoi/references.bib}

\end{document}